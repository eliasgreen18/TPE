\documentclass[a4paper]{article}
\input{Algo1Macros}

\usepackage{a4wide}
\usepackage{amsmath, amscd, amssymb, amsthm, latexsym}
\usepackage[spanish,activeacute]{babel}
\usepackage{enumerate}

\setlength{\parskip}{0.1em}
\usepackage{caratula} % Version modificada para usar las macros de algo1 de ~> https://github.com/bcardiff/dc-tex

\begin{document}
	
	\titulo{TP de Especificación}
	\subtitulo{Juego de la vida toroidal}
	\fecha{\today}
	\materia{Algoritmos y Estructuras de Datos I}
	\grupo{Grupo: Compubiólogos}
	
	\newcommand{\senial}{\textit{se\~nal}}
	
	\integrante{Cerdeira, Elías Nahuel}{692/12}{ecerdeira@dc.uba.ar}
	\integrante{Rodriguez Ferrante, Guadalupe}{60/12}{guadarodriguezf@gmail.com}
	
	\maketitle
	
	\section{Problemas}
    	\begin{proc}{esValido}{\In t: $toroide$, \Out result: $\bool$}{}
    		\pre{\True}
    		\post{result \iff valido(t) }
    	\end{proc} \\
	
    	\begin{proc}{posicionesVivas}{\In t: $toroide$, \Out result: \TLista{\ent\times\ent}}{}
    		\pre{valido(t)}
    		\post{\longitud{result}=cantVivas(t) \land noRepetidos(result) \land (\forall i:\ent) ((enRango(i,result) \implicaLuego (enRango(result[i][0],t) \land enRango(result[i][1],t[0]) ) \yLuego  estaViva(t[result[i][0]][result[i][1]])))}
	    \end{proc} \\
	
	    \begin{proc}{densidadPoblacion}{\In t: $toroide$, \Out result: \rea}{}
		    \pre{valido(t)}
		    \post{result = cantVivas(t)/posicionesTotales(t)}
	    \end{proc} \\
	    
	    \begin{proc}{evolucionDePosicion}{\In t: $toroide$, \In posicion: \ent \times \ent, \Out result: $\bool$}{}
		    \pre{valido(t) \yLuego (enRango(posicion[0],t) \land enRango(posicion[1],t[0])}
		    \post{result = posicionModificada(t, posicion)}
	    \end{proc} \\
	    
	    \begin{proc}{evolucionToroide}{\Inout t: $toroide$}{}
		    \pre{valido(t)}
		    \post{(\forall i:\ent)(enRango(i,t) \implicaLuego (\forall j:\ent) (enRango(j,t[0]) \implicaLuego t[i][j]=posicionModificada(t, (i, j))))}
	    \end{proc} \\
	    
	    \begin{proc}{evolucionMultiple}{\Inout t: $toroide$, \In k: \ent, \Out result: $toroide$}{}
    		\pre{valido(t) \land esPositivo(k)}
    		\post{}
    	\end{proc} \\
    	
    	\begin{proc}{esPeriodico}{\In t: $toroide$, \Inout p: \ent, \Out result: \bool}{}
    		\pre{valido(t)}
    		\post{} % Recordar que debo especificar sobre qué sucede con p y con res;
    	\end{proc} \\
    	
    	\begin{proc}{primosLejanos}{\In t1: $toroide$, \In t2: $toroide$, out primos: \bool}{}
    		\pre{valido(t1) \land valido(t2)}
    		\post{}
    	\end{proc} \\
    	
    	\begin{proc}{seleccionNatural}{\In ts: \TLista{$toroide$}, \Out res: \ent}{}
    		\pre{valido(t)}
    		\post{}
    	\end{proc} \\
    	
    	\begin{proc}{fusionar}{\In t1: $toroide$, \In t2: $toroide$, \Out res: $toroide$}{}
    		\pre{valido(t1) \land valido(t2) \land mismasDimensiones(t1,t2)}
    		\post{(\forall i:\ent)(enRango(i,t1) \implicaLuego (\forall j:\ent)(enRango(j,t1[0]) \implicaLuego ( (estaViva(t1,i,j)) \land estaViva(t2,i,j) = estaViva(res,i,j))))}
    	\end{proc} \\
	
	    % Con la especificación de vistaTrasladada me quedan dudas si a y b son "existe". Por otro lado, ver cómo hacemos cuando i+a pasa el borde e idem con j+b. Pensando como en implementación (imposible no hacerlo) haría como una función que "desplace" el valor de la celda. Lo importante es que claramente está incompleta o directamente hay que reformular.
	
	    \begin{proc}{vistaTrasladada}{\In t1: $toroide$, \In t2: $toroide$, \Out res: \bool}{}
    		\pre{valido(t1) \land valido(t2) \land mismasDimensiones(t1,t2)}
    		\post{res \iff (\forall i : \ent)(enRango(i,t1) \implicaLuego (\forall j : \ent)(enRango(j,t1[0]) \implicaLuego (\exists a : \ent)(enRango(a,t1) \yLuego (\exists b : \ent)(enRango(b,t1[0]) \yLuego (estaViva(t1,i,j)=estaViva(t2,i+a,j+b) ) ) ) ) )}
    	\end{proc} \\
    	
    	\begin{proc}{enCrecimiento}{\In t: $toroide$, \Out res: \bool}{}
    		\pre{valido(t)}
    		\post{}
    	\end{proc} \\
	
	\section{Predicados y Auxiliares generales}
	
    	\subsection{Predicados}
	
        	\auxpred {valido}{t:$toroide$}{0<\longitud{t} \yLuego \paraTodo{i, j}{0}{\longitud{t}} \implicaLuego (\longitud {t[i]}= \longitud{t[j]} \land \longitud {t[i]}>0)} \\
	        
	        \auxpred{enRango}{i:\ent, s:\TLista{T}}{ $0 \leq i < \longitud{s}$} \\
	        
	        \auxpred{estaViva}{t:$toroide$, i:\ent, j:\ent}{t[i][j]=\True} \\
	
        	\auxpred {noRepetidos}{t:$toroide$}{\paraTodo{i,j}{(0}{\longitud{t} \land i\ne j) \implicaLuego t[i] \ne t[j]}} \\

        	\auxpred{posicionModificada}{t: $toroide$, posicion: \ent \times \ent}{\IfThenElse{estaViva(t,posicion[0],posicion[1])}{siVivaLuego(t, posicion)}{siMuertaLuego(t, posicion)}} \\
        	
        	\auxpred{siVivaLuego}{t: $toroide$, posicion: \ent \times \ent}{2 \leq cantVecinasVivas(t, posicion) \leq 3} \\
        	
        	\auxpred{siMuertaLuego}{t: $toroide$, posicion: \ent \times \ent}{cantVecinasVivas(t, posicion)=3} \\
        	
        	\auxpred{esVecina}{casilla : (\ent \times \ent), vecina : (\ent \times \ent), t: $toroide$ }{casilla \ne vecina \yLuego (esVecinaEnFila(casilla, vecina, t) \oLuego esVecinaEnColumna(casilla, vecina, t) \oLuego esVecinaEnDiagonalDer(casilla, vecina, t) \oLuego esVecinaEnDiagonalIzq(casilla, vecina, t)))} \\
        	
        	\auxpred{esVecinaEnFila}{casilla : (\ent \times \ent), vecina : (\ent \times \ent), t: $toroide$ }{casilla[0]=vecina[0] \yLuego ((casilla[1]=\longitud{t[0]}-1 \land (vecina[1]=0 \lor vecina[1]=\longitud{t[0]}-2)) \lor (casilla[1]=0 \land (vecina[1]=\longitud{t[0]}-1 \lor vecina [1]=1)) \oLuego ( (vecina[1]=casilla[1]-1) \lor vecina[1]=casilla[1]+1))} \\
        	
        	\auxpred{esVecinaEnColumna}{casilla : (\ent \times \ent), vecina : (\ent \times \ent), t: $toroide$ }{casilla[1]=vecina[1] \yLuego ((casilla[0]=\longitud{t}-1 \land (vecina[0]=0 \lor vecina[0]=\longitud{t}-2)) \lor (casilla[0]=0 \land (vecina[0]=\longitud{t}-1 \lor vecina [0]=1)) \oLuego ( (vecina[0]=casilla[0]-1) \lor vecina[0]=casilla[0]+1))} \\

        	\auxpred{esVecinaEnDiagonalDer}{casilla : (\ent \times \ent), vecina : (\ent \times \ent), t: $toroide$ }{(casilla[1]=0 \land vecina[1]= \longitud{t[0]}-1 \yLuego ((casilla[0]= 0 \land (vecina[0]= \longitud{t}-1 \oLuego vecina[0]=1)) \lor (casilla[0]=\longitud {t}-1 \land (vecina[0]=0 \oLuego vecina[0]=\longitud{t}-2)) \oLuego (vecina[0]=casilla[0]-1 \lor vecina[0]=casilla[0]+1))) \oLuego (casilla[1]\ne 0 \yLuego vecina[1]=casilla[1]-1 \land (vecina[0]=casilla[0]-1 \lor vecina[0]=casilla[0]+1))} \\

        	\auxpred{esVecinaEnDiagonalIzq}{casilla : (\ent \times \ent), vecina : (\ent \times \ent), t: $toroide$ }{(casilla[1]=\longitud{t[0]}-1 \land vecina[1]= 0 \yLuego ((casilla[0]= 0 \land (vecina[0]= \longitud{t}-1 \oLuego vecina[0]=1)) \lor (casilla[0]=\longitud {t}-1 \land (vecina[0]=0 \oLuego vecina[0]=\longitud{t}-2)) \oLuego (vecina[0]=casilla[0]-1 \lor vecina[0]=casilla[0]+1))) \oLuego (casilla[1] \ne \longitud{t[0]}-1 \yLuego vecina[1]=casilla[1]+1 \land (vecina[0]=casilla[0]-1 \lor vecina[0]=casilla[0]+1))} \\
        	
        	\auxpred{esPositivo}{n:\ent}{0 \leq n} \\
        	
        	\auxpred{mismasDimensiones}{t1: $toroide$, t2: $toroide$}{mismaCantidadFilas(t1,t2) \land  mismaCantidadColumnas(t1,t2)} \\
        	
        	\auxpred{mismaCantidadFilas}{t1: $toroide$, t2: $toroide$}{cantidadFilas(t1) = cantidadFilas(t2)} \\
        	
        	\auxpred{mismaCantidadColumnas}{t1: $toroide$, t2: $toroide$}{cantidadColumnas(t1) = cantidadColumnas(t2)} \\
	
    	\subsection{Funciones Auxiliares}
	
      	\aux{cantVivas}{t:$toroide$}{\ent}{\sum_{i=0}^{\longitud{t}-1}\sum_{j=0}^{\longitud{t[i]}-1}\IfThenElse{estaViva(t,i,j)}{1}{0}}
      	
      	\aux{cantVecinasVivas}{t:$toroide$, posicion :\ent \times \ent}{\ent}{\sum_{i=0}^{\longitud{t}-1}\sum_{j=0}^{\longitud{t[i]}-1}\IfThenElse{(esVecina(posicion, (i, j), t) \land estaViva(t,i,j))}{1}{0}}
	
        \aux{posicionesTotales}{t:$toroide$}{\ent}{\longitud{t} \times \longitud{t[0]}}	% Podría ser una auxiliar que arroje la cantidad de celdas posicionesTotales(t); Lo pense pero es super simple la cuenta y no se si nos sirve para mucho. pero me da igual.
        
        \aux{cantidadFilas}{t:$toroide$}{\ent}{\longitud{t}}
        
        \aux{cantidadColumnas}{t:$toroide$}{\ent}{\longitud{t[0]}}
	
	
	
	\section{Decisiones tomadas}
	% Sólo para decisiones de alto nivel, no para explicar la solución a los ejercicios: CONSULTAR!
	
	% Incluso, para quedar más "legible" podemos poner un auxiliar "cantidad de filas" y otro "cantidad de columnas" o algo similar.
	
\end{document}
