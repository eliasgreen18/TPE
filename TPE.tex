\documentclass[a4paper]{article}
\input{Algo1Macros}

\usepackage{a4wide}
\usepackage{amsmath, amscd, amssymb, amsthm, latexsym}
\usepackage[spanish,activeacute]{babel}
\usepackage{enumerate}

\setlength{\parskip}{0.1em}
\usepackage{caratula} % Version modificada para usar las macros de algo1 de ~> https://github.com/bcardiff/dc-tex

\begin{document}
	
	\titulo{TP de Especificación}
	\subtitulo{Juego de la vida toroidal}
	\fecha{\today}
	\materia{Algoritmos y Estructuras de Datos I}
	\grupo{Grupo: Compubiólogos}
	
	\newcommand{\senial}{\textit{se\~nal}}
	
	\integrante{Cerdeira, Elías Nahuel}{692/12}{ecerdeira@dc.uba.ar}
	\integrante{Rodriguez Ferrante, Guadalupe}{60/12}{guadarodriguezf@gmail.com}
	
	\maketitle
	
	\section{Problemas}
    	\begin{proc}{esValido}{\In t: $toroide$, \Out result: $\bool$}{}
    		\pre{\True}
    		\post{result \iff valido(t) }
    	\end{proc} \\
	
    	\begin{proc}{posicionesVivas}{\In t: $toroide$, \Out result: \TLista{\ent\times\ent}}{}
    		\pre{valido(t)}
    		\post{\longitud{result}=cantVivas(t) \land noRepetidos(result) \land \paraTodo{i}{(0}{\longitud{result}-1 \implicaLuego(( 0 \leq result[i][0] < \longitud{t}-1) \land(  0 \leq result[i][1] < \longitud{t[0]}-1)) \yLuego (t[result[i][0]][result[i][1]]= \True))}}
	    \end{proc} \\
	
	    \begin{proc}{densidadPoblacion}{\In t: $toroide$, \Out result: \rea}{}
		    \pre{valido(t)}
		    \post{result = cantVivas(t)/(\longitud {t}\times \longitud{t[0]})}
	    \end{proc}
	
	
	
	\section{Predicados y Auxiliares generales}
	
    	\subsection{Predicados}
	
        	\auxpred {valido}{t:$toroide$}{\longitud{t}>0 \yLuego \paraTodo{i, j}{0}{\longitud{t}} \implicaLuego (\longitud {t[i]}= \longitud{t[j]} \land \longitud {t[i]}>0)} \\
	        
        	% Predicado enRango(i,s);
	
        	% Podría ser un predicado estaViva(i,j,t);
	
        	\auxpred {noRepetidos}{t:$toroide$}{\paraTodo{i,j}{(0}{\longitud{t} \land i\ne j) \implicaLuego t[i] \ne t[j]}} \\
        	
        	\auxpred{esToroidePequeno}{t: $toroide$ }{\longitud{t} \leq 3 \land \longitud{t[0]} \leq 3} \\
        	
        	\auxpred{esVecinaEnFila}{casilla : (\ent \times \ent), vecina : (\ent \times \ent), t: $toroide$ }{casilla[0]=vecina[0] \land ((casilla[1]=\longitud{t[0]}-1 \land (vecina[1]=0 \lor vecina[1]=\longitud{t[0]}-2)) \lor (casilla[1]=0 \land (vecina[1]=\longitud{t[0]}-1 \lor vecina [1]=1)) \oLuego ( (vecina[1]=casilla[1]-1) \lor vecina[1]=casilla[1]+1))} \\
        	
        	\auxpred{esVecinaEnColumna}{casilla : (\ent \times \ent), vecina : (\ent \times \ent), t: $toroide$ }{casilla[1]=vecina[1] \land ((casilla[0]=\longitud{t}-1 \land (vecina[0]=0 \lor vecina[0]=\longitud{t}-2)) \lor (casilla[0]=0 \land (vecina[0]=\longitud{t}-1 \lor vecina [0]=1)) \oLuego ( (vecina[0]=casilla[0]-1) \lor vecina[0]=casilla[0]+1))} \\

        	\auxpred{esVecinaEnDiagonalDer}{casilla : (\ent \times \ent), vecina : (\ent \times \ent), t: $toroide$ }{(casilla[1]=0 \land vecina[1]= \longitud{t[0]}-1 \land ((casilla[0]= 0 \land (vecina[0]= \longitud{t}-1 \lor vecina[0]=1)) \lor (casilla[0]=\longitud {t}-1 \land (vecina[0]=0 \lor vecina[0]=\longitud{t}-2)) \oLuego (vecina[0]=casilla[0]-1 \lor vecina[0]=casilla[0]+1))) \oLuego (vecina[1]=casilla[1]-1 \land (vecina[0]=casilla[0]-1 \lor vecina[0]=casilla[0]+1))} \\

        	\auxpred{esVecinaEnDiagonalIzq}{casilla : (\ent \times \ent), vecina : (\ent \times \ent), t: $toroide$ }{(casilla[1]=\longitud{t[0]}-1 \land vecina[1]= 0 \land ((casilla[0]= 0 \land (vecina[0]= \longitud{t}-1 \lor vecina[0]=1)) \lor (casilla[0]=\longitud {t}-1 \land (vecina[0]=0 \lor vecina[0]=\longitud{t}-2)) \oLuego (vecina[0]=casilla[0]-1 \lor vecina[0]=casilla[0]+1))) \oLuego (vecina[1]=casilla[1]+1 \land (vecina[0]=casilla[0]-1 \lor vecina[0]=casilla[0]+1))} \\
        	
        	\auxpred{esVecina}{casilla : (\ent \times \ent), vecina : (\ent \times \ent), t: $toroide$ }{(esToroidePequeno (t) \land (\longitud{t}>1\land \longitud {t[0]-1}>1) \land casilla \ne vecina) \lor (\neg (esToroidePequeno(t)) \yLuego (esVecinaEnFila(casilla, vecina, t) \lor esVecinaEnColumna(casilla, vecina, t) \lor esVecinaEnDiagonalDer(casilla, vecina, t) \lor esVecinaEnDiagonalIzq(casilla, vecina, t)))} \\
	
    	\subsection{Funciones Auxiliares}
	
      	\aux{cantVivas}{t:$toroide$}{\ent}{\sum_{k=0}^{\longitud{t}-1}\sum_{j=0}^{\longitud{t[k]}-1}\IfThenElse{(t[k][j]=\True)}{1}{0}}
      	
      	\aux{cantVecinasVivas}{t:$toroide$, posicion :\ent \times \ent}{\ent}{\sum_{k=0}^{\longitud{t}-1}\sum_{j=0}^{\longitud{t[k]}-1}\IfThenElse{(esVecina(posicion, (j, k), t) \land t[k][j]=\True)}{1}{0}}
	
        	% Podría ser una auxiliar que arroje la cantidad de celdas posicionesTotales(t);
	
	
	
	\section{Decisiones tomadas}
	% Sólo para decisiones de alto nivel, no para explicar la solución a los ejercicios: CONSULTAR!
	
	% Decimos lo que pensamos hacer con respecto a ser vecino de uno mismo.
	
\end{document}
